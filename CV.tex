\documentclass[11pt, a4paper]{moderncv}
\moderncvstyle{classic}
\moderncvcolor{blue}

\usepackage[utf8]{inputenc}
\usepackage[top=1.1cm, bottom = 1.1cm, left = 2cm, right = 2cm]{geometry}
%largeur colonnes pour dates
\setlength{\hintscolumnwidth}{2.5cm}

%entete
\firstname{Gilles}
\familyname{Major}
\title{Développeur mobile iOS / Android}
\address{376, chaussée de Roodebeek}{1200 Bruxelles}{Belgique}
\email{major.gilles@gmail.com}
\mobile{0495 26 06 83}


\begin{document}
\makecvtitle
\section{Expérience professionnelle}
	\cventry{Septembre 2014 -- \linebreak Octobre 2014}
		{Développeur mobile (stagiaire)}
		{Belighted}
		{Louvain-La-Neuve}
		{}
		{
			\begin{itemize}
				\item réalisation d`un client mobile pour le CRM Scale
				\item récupération des données via RestKit
				\item caching local via CoreData
				\item synchronisation des données avec l’application web
				\item développement d’une interface conviviale en coopération avec un designer
			\end{itemize}
			}
	\cventry
		{Aout 2012  -- \linebreak Juillet 2013}
		{User Experience Coordinator}
		{OpenERP (maintenant Odoo)}
		{Grand-Rosière-Hottomont}
		{}
		{
			\begin{itemize}
				\item tests d’usabilité manuels et avec participants
				\item gestion de projet en collaboration avec une équipe de développeurs
				\item conseil fonctionnel  auprès des développeurs
				\item création de mockups d’interfaces graphiques
				\item bug reporting et création des spécifications
			\end{itemize}
			}
	
	\cventry
		{2011 -- \linebreak Aout 2012}
		{Missions intérimaires}
		{Randstad}
		{Bruxelles}
		{}
		{
			\begin{itemize}
				\item travail administratif
				\item gestion de bases de données
			\end{itemize}			
			}
	
	\cventry
		{2010}
		{Psychologue du travail (stagiaire)}
		{Cubiks}
		{Malines}
		{}
		{
			\begin{itemize}
				\item référentiel de compétences Cubiks
				\item création de  communication packages
				\item construction de tests psychotechniques
				\item traductions: outils psychologiques, marketing de la société
			\end{itemize}
			}
	\cventry
		{2001--2002}
		{Database-coordinator}
		{WorldCom S.A.}
		{Diegem}
		{}{}
\section{Etudes et formations}
	\cventry{2014}
		{Développeur d’applications mobiles iOS-Android}
		{Bruxelles Formation}
		{}
		{}
		{}
	\cventry
		{2013 -- \linebreak aujourd'hui}
		{Bachelier en Informatique de Gestion(en cours)}
		{ISFCE}
		{Bruxelles}
		{}
		{}
	\cventry
		{2010}
		{Master en Psychologie Industrielle, Économique et des 
			Organisations}
		{Université libre de Bruxelles}
		{Bruxelles}
		{Distinction}
		{}

	\cventry
		{2009}
		{Master en Sciences Psychologiques et de l’Education 
			(neuropsychologie)}
		{Université libre de Bruxelles}
		{Bruxelles}
		{Distinction}
		{} 
	\cventry
		{2007}
		{Bachelier en Sciences Psychologiques et de l’Education}
		{Université libre de Bruxelles}
		{Bruxelles}
		{Distinction}
		{}
	\cventry
		{2003}
		{Candidature en Philologie germanique (Anglais et Allemand)}
		{Université libre de Bruxelles}
		{Bruxelles}
		{Satisfaction}
		{}

\section{Connaissances techniques}
	\cvitem
		{Languages de programmation}
		{
			\begin{itemize}
				\item Avance: Objective-C
				\item Intermediaire: Java, C
				\item Debutant: Swift
			\end{itemize}
			}
	\begin{itemize}
		\item Langages de programmation: C, Java, Objective-C, Swift
		\item Frameworks: iOS SDK, Android SDK, Core Data, RestKit, AFNetworking
		\item Autres langages: bash, python, SQL
		\item Services web: parsing JSON et XML,  RESTful web services
		\item Modélisation: Merise, UML, mockups UI, flowcharts
		\item Système de contrôle de version: git, github
		\item Suivi de projet: Pivotal, Trello
		\item Outils d'édition: Appcode, IntelliJ, Android Studio, Eclipse, Vim, XCode, Sublime Text
		\item Connaissances fonctionnelles en logiciels ERP et flux d’entreprise
		\item Analyse des besoins fonctionnels et non fonctionnels
		\item Tests logiciels : tests manuels, bug reporting, scénarios de test
	\end{itemize}

\section{Atouts personnels}
	\begin{itemize}
		\item Autonome, \textbf{autodidacte}, curieux
		\item \textbf{Résout efficacement} les problèmes
		\item Goût du \textbf{challenge}
		\item Teamplayer, capacité d’écoute
		\item Flexible tout en étant structuré
	\end{itemize}


		
\section{Langues}
	\begin{itemize}
		\item Français : langue maternelle
		\item Anglais : lu, écrit et parlé, très bon niveau 
		\item Allemand : bonne compréhension de la langue écrite et parlée
		\item Néerlandais : bonne compréhension de la langue écrite et parlée 
	\end{itemize}

\section{Réalisations}
	\cvitem
		{Scale(CRM)}
		{
			\begin{itemize}
				\item gestion des opportunités, contacts, factures et flux d’activités
				\item Synchronisation à double sens entre les clients mobiles et le serveur
			\end{itemize} 
		}
	\cvitem
		{VetoDico}
		{
			\begin{itemize}
				\item client Android d’un dictionnaire vétérinaire (3000+ définitions) avec historique des recherches, gestion des favoris et SyncAdapter
 				\item web service python realisé avec Flask et SqlAlchemy
			\end{itemize}
			}

\section{Centres d’intérêt et loisirs}
		Lecture, musique, programmation, technologie
\end{document}